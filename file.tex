\documentclass[a4paper, 12pt]{article}
\usepackage[T2A]{fontenc} 
\usepackage[utf8]{inputenc} 
\usepackage{amssymb}
\usepackage{amsmath}
\usepackage[english, russian]{babel} 
\setlength{\parskip}{1.3ex}
\DeclareMathOperator*{\sign}{sign}

\begin{document}
\begin{center} \huge
Решение системы уравнений
\end{center}
\textbf{
Задача 21.
}
Найти все (x, y) $\in \mathbb{R}^2 $, удовлетворяющие системе:

$$
\left\{
\begin{array} {l c l r}
4y + \frac{1}{1+\th^2{x}} + \frac{1}{1+\th^2{y}} &= &\alpha &\\
&&& (1) \\
\cos{y} + 2x + \frac{1}{2}\arctg{x}&= &0&
\end{array}
\right.
$$
\newline
\begin{text} \bf
\textbf{
Теорема.
}
Если верны следующие условия:
\end{text}
\begin{enumerate}
\item $X$ - полное нормированное пространство
\item $\Omega \subset X$ - выпуклое множество
\item $\Omega$ замкнуто
\item Отображение $ P:\Omega\times A \to \Omega$ 
\item $\forall a \in A, \: \forall x \in \Omega \;\; \exists P^{'}_x(x, \alpha)$
\item $\exists q \in (0, 1): \; \underset{x \in \Omega}{\sup} ||P^{'}_x(x, \alpha)|| \leq q < 1, \;\forall a \in A$
\item $\forall x \in \Omega$ отображение $P(x, \alpha)$ непрерывно по $\alpha$ в точке $\alpha_0$
\end{enumerate}
\text{
То:
}
\begin{enumerate}
\item $\forall \alpha \in A$ $\exists!$ решение $x = x(\alpha)$ уравнения $P(x_*, \alpha) = x_*$
\item $x(\alpha) = \underset{{n\to \infty}}{\lim}x_n(\alpha)$ и $x_{n+1}(\alpha) = P(x_n(\alpha), \alpha)$
\item Скорость сходимости: $\rho(x_n, x_*) \le \frac{q^{n}}{1-q} \rho(x_1, x_0)$
\item Отображение $x(\alpha)$ непрерывно в точке $\alpha_0$
\end{enumerate}

\begin{text} \bf
\newline
\textbf{Решение.}
\end{text}
Преобразуем систему (1):
$$
\left\{
\begin{array} {l c l r}
x &= & - \frac{1}{2}(\cos{y}+\frac{1}{2}\arctg{x}) &\\
&&& (2) \\
y &= &\frac{1}{4}(\alpha - \frac{1}{1+\th^2{x}} - \frac{1}{1+\th^2{y}})&
\end{array}
\right.
$$

Рассмотрим отображение $P: \Omega\times$A$\to\Omega,$
$$
((x, \: y), \: \alpha) \mapsto
\left(
\begin{array} {l c l}
- \frac{1}{2}(\cos{y}+\frac{1}{2}\arctg{x}) \\\\
\frac{1}{4}(\alpha - \frac{1}{1+\th^2{x}} - \frac{1}{1+\th^2{y}})
\end{array}
\right)
$$

Проверим выполнение условий теоремы.
В качестве множества $X$ возьмем $\mathbb{R}$. Теперь оценим $\Omega$:
$|\frac{1}{2}(\cos{y} + \frac{1}{2}\arctg{x})| < 1$ => |x| < 1, 
из первого ур-ия системы (1): $ -2 < \alpha < 6$,
тогда $|y| < 2$.
Таким образом, положив $\Omega = \left[-1; 1\right]\times\mathbb{R}$,
$ $A$ = \mathbb{R}$, получим, что условия 1-4 теоремы выполнены. 
$$
\forall \alpha \in A \;\;
P^{'}_x((x, y), \alpha) =
\left(
\begin{array}{cc}
-\frac{1}{4}\frac{1}{1+x^2} & \frac{\sin{y}}{2} \\\\
\frac{\th{2x}}{4\ch^2{2x}} & \frac{\th{2y}}{4\ch^2{2y}}
\end{array}
\right)
$$
Т.е. условие 5 выполнено. Т.к. $P$ линейна по $\alpha$, то условие 7 выполнено. Найдем теперь $q \in (0, 1): \; \underset{x \in \Omega}{\sup} ||P^{'}_x(x, \alpha)|| \leq q < 1, \;\forall a \in A$.
\newline
$\forall (x, y) \in \mathbb{R}^2: \;$
$$
\bigg|\frac{1}{4}\frac{1}{1+x^2}\bigg| \leq \frac{1}{4}, \;\;\;
\bigg|\frac{\sin{y}}{2}\bigg| \leq \frac{1}{2}
$$
$$
\left(\frac{\th{2x}}{4\ch^2{2x}}\right)' = \frac{3-\ch{4x}}{4\sh^3{2x}}, \;
x = \pm\frac{1}{4\ch{3}} \text{ - точки экстремума}
; \text{ покажем, что } \bigg|\frac{\th{2x}}{4\ch^2{2x}}\bigg| \leq \frac{1}{8}
$$
$$
19.8 \le e^3 \le 20.1 => 
19.85 \le e^3 + e^{-3} \le 20.15 =>
0.049 \le |2x| \le 0.05
$$
$$
=> 1.05 \le e^{2x} \le 1.052,
0.951 \le e^{-2x} \le 0.953 =>
2.001 \le e^{2x} + e^{-2x} \le 2.005,
$$
$$
0.98 \le e^{2x} - e^{-2x} \le 0.99,
\frac{\th{2x}}{4\ch^2{2x}} = \frac{\sh{2x}}{4\ch^3{2x}} = 
\frac{e^{2x} - e^{-2x}}{(e^{2x} + e^{-2x})^3} \le 0.124 \le \frac{1}{8}
$$
Рассмотрим ||$B$||:
$$
||B|| = \underset{j}\max\sum_i{|b_{ij}|}, \;\;
B = \left(
\begin{array}{cc}
\frac{1}{4} & \frac{1}{2} \\
\frac{1}{8} & \frac{1}{8}
\end{array}
\right)
=> ||B|| = \frac{1}{2} + \frac{1}{8} = \frac{5}{8} < 1
$$
Итак, все условия теоремы выполнены. Вычислим оценку числа шагов.
По теореме (п. 3): $\rho(x_n, x_*) \le \frac{q^{n}}{1-q} \rho(x_1, x_0)$.
Тогда с учетом $\rho(x_n, x_*) = \varepsilon,\; x_0 = (0, 0) $  получим:
$$
n \leq \log_q{\frac{\varepsilon(1-q)}{\rho(x_1, x_0)}} \;\;,
x_1 = \left(-\frac{1}{2}, \frac{\alpha-2}{4}\right), \;\;
\rho(x_1, x_0) = \frac{\sqrt{\alpha^2-4\alpha+8}}{4} $$
$$
n \leq \log_{\frac{5}{8}}{\frac{3\varepsilon}{2\sqrt{\alpha^2-4\alpha+8}}}
$$
Например, для $\varepsilon = 10^{-18},\; \alpha = 0: \; n \leq 89.5$

\end{document}
